%=======================================================================
% Manual concerning the format of the hamiltonian files.
%=======================================================================
%
%= Document class, packages and commands ===============================
%
\documentclass[a4paper,11pt]{article}
%
\usepackage[left=2.0cm,right=2.0cm,top=2.0cm,bottom=2.0cm,includefoot,includehead,headheight=13.6pt]{geometry}
\usepackage[T1]{fontenc}
\usepackage[utf8]{inputenc}
\usepackage{amsmath}
\usepackage[colorlinks,urlcolor=blue,linkcolor=blue]{hyperref}

\newcommand{\TAURUSvap}{$\text{TAURUS}_{\text{vap}}$}
\newcommand{\TAURUSvapt}{$\text{TAURUS}_{\text{vap}}$~}
\renewcommand{\tt}[1]{\texttt{#1}}

\newcommand{\bra}[1]{\langle #1 \vert}
\newcommand{\ket}[1]{\vert #1 \rangle}
\newcommand{\elma}[3]{\bra{#1} #2 \ket{#3}}

% 
% 
%= Begin document ======================================================
%
\begin{document}

%
%= Title =======================================
% 
\begin{center}
 {\LARGE \textbf{\TAURUSvap: Manual for the Hamiltonian files}} \\[0.20cm]
 {\large 05/02/2023}
\end{center}

%
%= Section: Types of Hamiltonian =======================================
% 
\section{Types of Hamiltonian}

The code \TAURUSvapt requires as input one or several files containing the informations about the model space and the Hamiltonian matrix elements.
The precise number of files required depends on the type of Hamiltonian considered as well as the parameters entered in the STDIN file.
One crucial parameter is the name of the Hamiltonian file, stored in the variable \tt{hamil\_file}, that determines the name of the files that the code will open.
As of today, the code supports 4 types of Hamiltonian associated in the code with the variable \tt{hamil\_type}:

\begin{itemize}
  \item \tt{hamil\_type = 1 or 2}: valence-space Hamiltonian in $JT$-scheme written in the format of the shell-model code ANTOINE.
  In this case, the code will require only one Hamiltonian file containing the model space, the single-particle energies and the 
  two-body matrix elements. 
  In the case \tt{hamil\_type = 1}, the single-particle energies of protons and neutrons are the same, while for 
  \tt{hamil\_type = 2} they are different. \\
  Naming convention for the files: \tt{hamil\_file.sho}
  
  \item \tt{hamil\_type = 3}: general Hamiltonian in $J$-scheme. In this case, the code will require
  3 files: 1 file containing the model space, 1 file containing the zero- and one-body parts of the Hamiltonian and 1 file containing the
  two-body part of the Hamiltonian. This is the most general type of Hamiltonian considered that allows one to define all the matrix elements
  manually. \\
  Naming convention for the files: \tt{hamil\_file.sho}, \tt{hamil\_file.01b}, \tt{hamil\_file.2b} 

  \item \tt{hamil\_type = 4}: ``bare'' Hamiltonian in $J$-scheme. In this case, the code will require
  2 files: 1 file containing the model space and 1 file containg two-body part of the Hamiltonian. For such Hamiltonians, the zero-body part
  is assumed to be equal to zero and the one-body part is built from the kinetic-energy one-body operator $T$. \\
  Naming convention for the files: \tt{hamil\_file.sho}, \tt{hamil\_file.2b} 
  
\end{itemize}

\subsubsection*{Center-of-mass correction}
For \tt{hamil\_type = 3 or 4}, it is possible to add a center-of-mass correction by setting \tt{hamil\_com = 1} in the
STDIN file. In that case, the code will require an additional file containing the two-body matrix elements for the center-of-mass correction.
The one-body part is automatically computed as $-\frac{T}{A}$ taking $A$ equal to the average number of nucleons targeted. \\
Naming convention for the files: \tt{hamil\_file.com} 

\subsubsection*{Reduced format}
For all types of Hamiltonian, it is possible to read the matrix elements  in a so-called ``reduced'' format from a binary file.
This is is done by setting \tt{hamil\_read = 1} in the STDIN file. In that case, the code will \emph{always} require 2 files: 1 file containing
the model space (\tt{hamil\_file.sho}) and 1 file containing all the matrix elements of the Hamiltonians in the Spherical Harmonic 
Oscillator (SHO) single-particle basis written in binary (more compact and faster to read). 
While the reduced format will not be explained here,\footnote{We provide in the directory \tt{extras/codes} a small code that can write Hamiltonians in the reduced format.} 
a reduced file will automatically be generated when using the code with \tt{hamil\_read = 0}. \\
Naming convention for the files: \tt{hamil\_file.red} 

%
%= Section: Antoine files ==============================================
%
\section{Valence-space $JT$-coupled Hamiltonian à la ANTOINE}

For those Hamiltonians, we follow almost exactly the format of the shell-model code ANTOINE as described on their
\href{http://www.iphc.cnrs.fr/nutheo/code\_antoine/menu.html}{website}. 
The only differences are:
\begin{itemize}
  \item The name of the interaction can accept up to 100 characters.
  \item The code does not handle different valence spaces for protons and neutrons (we need a space invariant under
  isospin rotation in order to be able to perform a subsequent isospin projection).
  \item It is possible to select the oscillator frequency $\hbar \omega$ by adding two parameters at the end of the line
  specifying the number of particles in the core. 
  For example, for the $sd$-shell: \tt{1 8 8 0.300 opt\_hw HO\_hw}. See next section for more details on these two parameters. 
  In case they are not specified, \tt{opt\_hw = 0} is assumed.
\end{itemize}
To learn more about this format, go visit the ``Manual'' section of their
\href{http://www.iphc.cnrs.fr/nutheo/code\_antoine/menu.html}{website}. 
 

%
%= Section: Vapes of Hamiltonian =======================================
%
\section{General $J$-coupled Hamiltonian}

\subsection{SHO model space (.sho)}

The file \tt{hamil\_file.sho} has the following format: 
\begin{center}
\begin{tabular}{|c|c|l|}
\hline
Line & \ Type \hfill & Data \\
\hline
 \textbf{1}   & character  & \tt{hamil\_name} \\
 \textbf{2}   & integer    & \tt{hamil\_type} \\
 \textbf{3}   & integers   & \tt{HOsh\_dim}  \: \tt{(HOsh\_na(i), i=1, HOsh\_dim)} \\
 \textbf{4}   & integers   & \tt{core\_Z} \: \tt{core\_N} \\
 \textbf{5}   & integer \: real  & \tt{opt\_hw} \: \tt{HO\_hw} \\
\hline
\end{tabular}
\end{center}
where in each line we have:
\begin{itemize}
%\noindent
\item[\textbf{1}] \tt{hamil\_name}: a name to describe the Hamiltonian. Only the first 100 characters will be stored.
\item[\textbf{2}] \tt{hamil\_type}: the type of Hamiltonian (see the first section). 
\item[\textbf{3}] \tt{HOsh\_dim}: the number of shells in the  model space. \\
                  \tt{(HOsh\_na(i), i=1, HOsh\_dim)}: loop to read the name of the shells. For a shell with quantum numbers $n,l,j$, the
                  name is $\tt{HOsh\_na} = 10000n + 100l + 2j$. Note that $n$ begins at 0. This is similar to the ANTOINE format except
                   that the first factor is 10000 (because $l$ can be larger than 10 in no-core calculations).
\item[\textbf{4}] \tt{core\_Z}: number of protons in the core. \\         
                  \tt{core\_N}: number of neutrons in the core.              
\item[\textbf{5}] \tt{opt\_hw}: option to control the value of the frequency for the Harmonic Oscillator. \\
                  \tt{= 0} the frequency is set using the formula $\hbar \omega = 45A^{-1/3} - 25A^{-2/3}$.\footnote{See equation (3.45) of the 
                  book \textit{From Nucleons to Nucleus} by J. Suhonen.} \\ 
                  \tt{= 1} the frequency is set using the formula $\hbar \omega = 41A^{-1/3}$.\footnote{See equation (3.44) of the 
                  book \textit{From Nucleons to Nucleus} by J. Suhonen.} \\ 
                  \tt{= 2} the frequency is set to the value \tt{HO\_hw}. \\
                  \tt{HO\_hw}: value of the frequency for the Harmonic Oscillator.
\end{itemize}

\subsubsection*{Example}
\begin{center}
\begin{tabular}{|l|}
\hline
  \tt{My Chiral EFT Hamiltionian for emax=4 and hw=20} \\
  \tt{3} \\
  \tt{15      1    103    101  10001    205    203  10103  10101    307    305  20001  10205  10203    409    407} \\
  \tt{0  0} \\
  \tt{20.0} \\
\hline
\end{tabular}
\end{center}

\subsection{Zero- and one-body parts (.01b)}

The file \tt{hamil\_file.01b}, read only if \tt{hamil\_type = 3}, has the following format: 
\begin{center}
\begin{tabular}{|c|c|l|}
\hline
Line & \ Type \hfill & Data \\
\hline
\textbf{1}    & character         & \tt{hname1} \\
\textbf{2}    & real              & \tt{hamil\_H0} \\
\textbf{3}    & int. int. real real & \tt{a} \:\tt{b} \:\tt{hamil\_H1cpd\_p} \:\tt{hamil\_H1cpd\_n} \\
\textbf{etc.} &     etc.          &        \phantom{000000} \tt{etc.}  \\
\hline
\end{tabular}
\end{center}
where in each line we have:
\begin{itemize}
\item[\textbf{1}] \tt{hname1}: dummy variable but has to be the \emph{same} name as \tt{hamil\_name}.
\item[\textbf{2}] \tt{hamil\_H0}: zero-body part $H^{0b}$ of the Hamiltonian.
\item[\textbf{3}] \tt{a}: left shell $a$ \\
                  \tt{b}: right shell $b$ \\
                  \tt{hamil\_H1cpd\_p}: one-body matrix element $\elma{a}{H^{1b}}{b}$ for proton shells. 
                  They will be attributed to all proton matrix elements $\elma{k}{H^{1b}}{l}$
                  between \emph{single-particle} states $k$ and $l$ that satisfy $k \in a$, $l \in b$ and $m_{j_k} = m_{j_l}$. \\
                  \tt{hamil\_H1cpd\_n}: same for neutrons.
\item[\textbf{etc.}] same for other combinations of shells.
\end{itemize}
Note that the order in which the matrix elements are written is not important, nor it is required to write all the matrix elements equal to zero.

\subsubsection*{Example}
\begin{center}
\begin{tabular}{|l|}
\hline
\tt{My Chiral EFT Hamiltionian for emax=4 and hw=20} \\
\tt{42.11792} \\
\tt{   \phantom{10}1      $\phantom{1000}$1   2.5330250        2.5675970} \\
\tt{   \phantom{10}1                  10001   5.5514280        5.5809940} \\
\tt{             103        \phantom{10}103   18.364205        18.366125} \\
        \phantom{00000000000} etc.  \\
\hline
\end{tabular}
\end{center}

\subsection{Two-body part (.2b)}

The file \tt{hamil\_file.2b} has the following format:
\begin{center}
\begin{tabular}{|c|c|l|}
\hline
Line & \ Type \hfill & Data \\
\hline
\textbf{1}   & character       & \tt{hname2} \\
\textbf{2}   & integers        & \tt{tmin} \:\tt{tmax} \:\tt{a} \:\tt{b} \:\tt{c} \:\tt{d} \:\tt{jmin} \:\tt{jmax} \\
\textbf{3}   & real            & \tt{(hamil\_H2cpd(t,j=jmin,a,b,c,d), t=tmin,tmax)} \\
$\vdots$     &   $\vdots$      &             \phantom{000000000000} $\vdots$ \\
\textbf{$\Delta = 3+\tt{jmax}-\tt{jmin}$}  & real   & \tt{(hamil\_H2cpd(t,j=jmax,a,b,c,d), t=tmin,tmax)} \\
  \textbf{etc.}       &     etc.        &        \phantom{00000000000}\tt{etc.}  \\
\hline
\end{tabular}
\end{center}
where in each line we have:
\begin{itemize}
\item[\textbf{1}] \tt{hname2}: dummy variable but has to be the \emph{same} name as \tt{hamil\_name}.
\item[\textbf{2}] \tt{tmin}: lowest isospin index. It has to be equal to 0. \\
                  \tt{tmax}: highest isospin index. It has to be equal to 5. \\
                  \tt{a}: left  shell $a$ \\
                  \tt{b}: left  shell $b$ \\
                  \tt{c}: right shell $c$ \\
                  \tt{d}: right shell $d$ \\
                  \tt{jmin}: minimum possible value for angular momentum coupling. \\
                  \tt{jmax}: maximum possible value for angular momentum coupling. 
\item[\textbf{3-$\Delta$}] \tt{hamil\_H2cpd(t,j,a,b,c,d)}: double loop over \tt{t=tmin,tmax} (horizontal) and \tt{j=jmin,jmax} (vertical) for the
                  matrix element $\elma{ab,jt}{H^{2b}}{cd,jt}$ in $J$-scheme. 
\item[\textbf{etc.}] same for other combinations of shells.
\end{itemize}
The value of the isospin index \tt{t} depends on the particle species (protons: $p$, neutrons: $n$) of the shells:
\begin{center}
\begin{tabular}{|c|c|c|c|c|}
\hline
 \tt{t} & $a$ & $b$ & $c$ & $d$ \\
\hline                   
 \tt{0} & $p$ & $p$ & $p$ & $p$ \\
 \tt{1} & $p$ & $n$ & $p$ & $n$ \\
 \tt{2} & $p$ & $n$ & $n$ & $p$ \\
 \tt{3} & $n$ & $p$ & $p$ & $n$ \\
 \tt{4} & $n$ & $p$ & $n$ & $p$ \\
 \tt{5} & $n$ & $n$ & $n$ & $n$ \\
\hline
\end{tabular}
\end{center}

Note that the order in which the \emph{blocks} of matrix elements for different values of (\tt{a}, \tt{b}, \tt{c}, \tt{d}) are written is not important, 
nor it is required to write a given block if all the matrix elements are zero. In addition, only one permutation of (\tt{a}, \tt{b}, \tt{c}, \tt{d}) 
is required.

\subsubsection*{Example}
\begin{center}
\begin{tabular}{|l|}
\hline
 \tt{My Chiral EFT Hamiltionian for emax=4 and hw=20} \\
 \tt{0   5       \phantom{10}1      \phantom{10}1      \phantom{1000}1      \phantom{1000}1    0   1} \\
 \tt{-8.23779E+00 -9.37161E+00 -9.37161E+00 -9.37161E+00 -9.37161E+00 -9.10953E+00} \\
 \tt{\phantom{-}0.00000E+00 -1.35045E+01 \phantom{-}1.35045E+01 \phantom{-}1.35045E+01 -1.35045E+01 \phantom{-}0.00000E+00} \\
 \tt{0   5     205    407  10103  10103    1   3} \\
 \tt{\phantom{-}0.00000E+00 -2.35514E-01 \phantom{-}2.35514E-01 \phantom{-}2.35514E-01 -2.35514E-01 \phantom{-}0.00000E+00} \\
 \tt{-2.76313E-01 -1.99398E-01 -1.99398E-01 -1.99398E-01 -1.99398E-01 -2.81405E-01} \\
 \tt{\phantom{-}0.00000E+00 \phantom{-}3.62107E-01 -3.62107E-01 -3.62107E-01 \phantom{-}3.62107E-01 \phantom{-}0.00000E+00} \\
        \phantom{00000000000} etc.  \\
\hline
\end{tabular}
\end{center}

\subsection{Center-of-mass correction (.com)}
The file containing the center-of-mass correction, \tt{hamil\_file.com}, has exactly the same format as the file containing 
the two-body part of the Hamiltonian. The only difference is that the first line containing the description of the file is never used
and can therefore be different from \tt{hamil\_name}. \\
Note that the matrix elements read will be multiplied by the factor $\frac{\hbar \omega}{A} \left( \frac{\hbar^2}{2 m} \right)^{-1}$.

%
%= End document ======================================================
%
\end{document}
%
%=====================================================================
